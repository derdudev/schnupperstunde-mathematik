\chapter{Lineare Gleichungssysteme}

Lösungsverfahren: Einsetzungs-, Gleichsetzungs- und Additionsverfahren

\section{Was macht ein Gleichungssystem linear?}

\begin{example}
   Existiert für das folgende lineare Gleichungssystem eine Lösung?

   (hässliches LGS einfügen, das aber homogen ist)
\end{example}

Beispiel dafür, dass man bei Wurzeln aufpassen sollte:
\begin{example}
    Wir betrachten die Gleichung 
    \begin{equation*}
        \sqrt{x+4} = -5
    \end{equation*}
    und wollen ihre reellen Lösungen bestimmen. Man könnte jetzt auf die Idee kommen, die Gleichung einfach auf beiden Seiten zu quadrieren und dann nach \(x\) umzustellen, also 
    \begin{align*}
        \sqrt{x+4} &= -5 && \mid (\dots)^2 \\
        (\sqrt{x+4})^2 &= (-5)^2 \\
        x+4 &= 25 && \mid -4 \\
        x &= 21 \\
    \end{align*}
\end{example}

\section{Warum sind LGS eigentlich so wichtig?}
Matrixschreibweise

Verbindung mit räumlicher Geometrie: Lösungsmengen von LGS sind Untervektorräume (Ebenen)!

\section{Das Gauß-Verfahren}
Systematisches Lösen von LGS 

Lösungen: Existenz und Eindeutigkeit

\section{Bestimmen von allgemeinen Funktionsgleichungen}
Große Beispielgallerie: Bestimmen von Polynomen oder ganz allgemein Funktionen mit Parameter anhand von Eigenschaften in Texten (siehe Schulklausur) oder Schaubildern. 