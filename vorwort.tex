\chapter{Vorwort}

\section{Für wen ist dieses Buch?}
Hallo, liebe Leserin, lieber Leser! Schön, dass du es bis in dieses Buch hinein in die Begrüßung geschafft hast. Test

Über \acrshort{gcd} oder \acrlong{lcm} sprechen wir später bestimmt auch noch. 

\section{Aufbau dieser "`Schnupperstunde"'}
Aufteilung der Kapitel in die Bereiche Analysis, Lineare Algebra, Geometrie und Stochastik/Statistik. 

Mit Matrizen kann hier auch gerechnet werden: 
\begin{equation}
    \begin{pNiceArray}{ccc|cc}[margin]
        1 & 2 & 3 & 1 & 0 \\
        4 & 5 & 6 & 0 & 1 \\
        \hline
        7 & 8 & 9 \\
        10 & 11 & 12
    \end{pNiceArray}, \quad 
    \begin{pNiceMatrix}[margin,hvlines]
        \Block{3-3}<\LARGE>{A} & & & 0 \\
        & \hspace*{1cm} & & \Vdots \\
        & & & 0 \\
        0 & \Cdots& 0 & 0
    \end{pNiceMatrix}, \quad 
    \begin{pNiceMatrix}
        I & 0 & \Cdots &0 \\
        0 & I & \Ddots &\Vdots\\
        \Vdots &\Ddots & I &0 \\
        0 &\Cdots & 0 &I
        \CodeAfter \line{2-2}{3-3}
    \end{pNiceMatrix}
\end{equation}
\NiceMatrixOptions{code-for-first-row = \scriptstyle,code-for-first-col = \scriptstyle }
\setcounter{MaxMatrixCols}{12}
\newcommand{\blue}{\color{blue}}
\begin{equation*}
    \begin{pNiceMatrix}[last-row,last-col,nullify-dots,xdots/line-style={dashed,blue}]
1& & & \Vdots & & & & \Vdots \\
& \Ddots[line-style=standard] \\
& & 1 \\
\Cdots & & & \blue 0 & \Cdots & & & \blue 1 & & & \Cdots & \blue \leftarrow i \\
& & & & 1 \\
& & &\Vdots & & \Ddots[line-style=standard] & & \Vdots \\
& & & & & & 1 \\
\Cdots & & & \blue 1 & \Cdots & & \Cdots & \blue 0 & & & \Cdots & \blue \leftarrow j \\
& & & & & & & & 1 \\
& & & & & & & & & \Ddots[line-style=standard] \\
& & & \Vdots & & & & \Vdots & & & 1 \\
& & & \blue \overset{\uparrow}{i} & & & & \blue \overset{\uparrow}{j} \\
\end{pNiceMatrix}
\end{equation*}
Beispiele von: https://mirror.funkfreundelandshut.de/latex/macros/latex/contrib/nicematrix/nicematrix.pdf
