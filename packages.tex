\usepackage[ngerman]{babel}

\usepackage{graphicx, float}
\usepackage[top=2.5cm, bottom=2.5cm, left=3.5cm, right=3.5cm]{geometry}

\usepackage[T1]{fontenc}
\usepackage{sourcecodepro}
% \usepackage{mathptmx}

\usepackage{lineno}
% \linenumbers

\usepackage{imakeidx}
\makeindex
% multiple indices: https://tex.stackexchange.com/questions/472/how-can-i-have-two-or-more-distinct-indexes

% to float text around figures better
\usepackage{wrapfig}

% to include extern pdfs: https://tex.stackexchange.com/questions/438775/how-to-insert-a-pdf-page-as-a-front-cover
\usepackage{pdfpages}

\usepackage{tikz}
\usepackage{pgfplots}
\pgfplotsset{compat=newest}
\usetikzlibrary{arrows.meta}
\usetikzlibrary{matrix, graphs}
\usetikzlibrary{hobby} % for blobs

% define custom lists
\usepackage[shortlabels]{enumitem}
\newlist{thmenum}{enumerate}{1} % to be used only inside 'theorem' environments
\setlist[thmenum]{label=\textup{(\arabic*)}} % , ref=\thetheorem \textup{(\arabic*)}}

\usepackage{hyperref}
\hypersetup{
    colorlinks=true,
    linkcolor=blue,
    filecolor=magenta,      
    urlcolor=blue,
    pdftitle={Overleaf Example},
    pdfpagemode=FullScreen,
    }

\usepackage{nicematrix}

%\usepackage{asymptote} 

\usepackage{amssymb}
\usepackage{amsfonts}
\usepackage{mathrsfs}
\usepackage{mathtools}

%\usepackage{minted}

\usepackage[capitalize]{cleveref}

\newcommand{\Z}{\mathbb{Z}}
\newcommand{\R}{\mathbb{R}}
\newcommand{\Imh}{\normalfont\text{Im } \Bbb H}
\newcommand{\dsum}{\oplus}
\newcommand{\normal}[1]{{\normalfont #1}}
\newcommand{\bild}{\normalfont\text{Bild}(\varphi)}
\newcommand{\bilda}{\normalfont\text{Bild}(\varphi_A)}
\newcommand{\bildb}{\normalfont\text{Bild}(\varphi_B)}
\newcommand{\rg}{\normalfont\text{rg}}
\newcommand{\gl}{\normalfont\text{GL}}
\newcommand{\kernm}{\normalfont\text{Kern}}
\newcommand{\spur}[1]{\,\normalfont\text{Spur}\, #1}
\newcommand{\real}[1]{\,\normalfont\text{Re}\, #1}
\newcommand{\eqv}{\:\Leftrightarrow\:}
\newcommand{\sprod}[1]{\langle #1 \rangle}
\newcommand{\ggT}{\normal{\text{ggT}}}
\newcommand{\diag}{\normal{\text{diag}}}

\DeclareMathOperator{\poisson}{Poisson}
\DeclareMathOperator{\binomial}{Binomial}

% see https://tex.stackexchange.com/questions/2705/typesetting-column-vector
\newcommand*\colvec[3][]{
\begin{pmatrix}\ifx\relax#1\relax\else#1\\\fi#2\\#3\end{pmatrix}
}

\renewcommand{\textbf}[1]{{\normalfont\bf #1}}

% für matrix gauß-operationen aka. pfeile: https://tex.stackexchange.com/questions/40280/how-can-i-visualize-matrix-operations
\usepackage{gauss} 

\usepackage[acronym]{glossaries}

\newacronym{gcd}{GCD}{Greatest Common Divisor}
\newacronym{lcm}{LCM}{Least Common Multiple}

\makeglossaries

\usepackage{amsthm}

\usepackage{newtxtext}        
\usepackage[varvw]{newtxmath}

\newtheoremstyle{mystyle}% name
  {.5em}% Space above
  {\topsep}% Space below
  {\itshape}% Body font
  {}% Indent amount
  {\bfseries}% Theorem head font
  {.}%Punctuation after theorem head
  {.5em}%Space after theorem head
  {}% theorem head spec
\theoremstyle{mystyle}

\newtheorem*{theorem*}{Satz}
%% this allows for theorems which are not automatically numbered

\newtheorem{theorem}{Satz}[chapter]
\newtheorem{definition}[theorem]{Definition}
\newtheorem{lemma}[theorem]{Lemma}
\newtheorem{corollar}[theorem]{Korollar}
\newtheorem{proposition}[theorem]{Proposition}
\newtheorem{exercise}[theorem]{Aufgabe}
\newtheorem{problem}[theorem]{Problem}
\newtheorem{annotation}[theorem]{Anmerkung}
\newtheorem{algorithm}[theorem]{Algorithmus}

\AfterEndEnvironment{definition}{\noindent\ignorespaces}

\theoremstyle{definition}
\newtheorem{exampleth}[theorem]{Beispiel}
\newtheorem{exkursth}[theorem]{Exkurs}

\newenvironment{exkurs}{\begin{exkursth}}{\hspace{\fill}$\triangleleft$\end{exkursth}}
\newenvironment{example}{\begin{exampleth}}{\hspace{\fill}$\triangleleft$\end{exampleth}}

\usepackage{lineno}
%% The above lines are for formatting.  In general, you will not want to change these.

\usepackage[utf8]{inputenc}

\numberwithin{equation}{chapter}

% exclude subsections from TOC: https://latex.org/forum/viewtopic.php?t=1309
\setcounter{tocdepth}{1}