\chapter{Integration}

\section{Das Riemann-Integral}
Intuitive Herleitung: Fläche unter Graphen. => Flächeninhalt auf verschiedene Arten abschätzen an konkreten Beispielen (Konstante Funktion, Gerade, Parabel, Gebrochenrationale Funktion: Problem => Was machen bei Polstellen?). Insbesondere Betrachtung durch Riemann vs Betrachtung durch Lebesgue. 

Einfaches Riemann-Integral Definition (auch Exkurs mit formaler Definition). 

Orientierter Flächeninhalt! => Große Beispiel Gallerie mit den verschiedenen Fällen

\section{Stammfunktionen}
"`Integration als Umkehrung von Ableiten"' => Rechtfertigen durch Stammfunktionen und Hauptsatz. => Fundamentaler Zusammenhang: Hauptsatz

Definition, Existenz, (Nicht-)Eindeutigkeit

\begin{definition}[Stammfunktion, Integrieren]
    Sei \(f\) eine Funktion. Eine auf einem offenen Intervall \((a,b)\) differenzierbare Funktion \(F\) heißt dann \textbf{Stammfunktion} von \(f\), falls \(F' = f\). 
\end{definition}

Stammfunktionen sind \textit{garantiert nie} eindeutig!
\begin{example}[Nicht-Eindeutigkeit von Stammfunktionen]
    
\end{example}

Warum sind Stammfunktionen relevant? => Hauptsatz der Differential- und Integralrechnung. 

\begin{theorem}[Hauptsatz der Differential- und Integralrechnung]
    
\end{theorem}

\section{Integrationsregeln}
Ableiten ist einfach: Man kann konkret den Differenzenquotienten betrachten oder einfacher mit den Ableitungsregeln arbeiten. Integrieren ist da um einiges schwieriger. Hier muss man sehr viel raten. 

Großer Satz: Potenzregel, Faktorregel, Summenregel, Lineare Substitution. 
\begin{theorem}[Integrationsregeln]
    
\end{theorem}

Besondere Stammfunktionen: 1/x, sin, cos, exp, log

\begin{theorem}[Rechenregeln für Integrale]
    
\end{theorem}

\section{Integralfunktionen}

\begin{definition}[Integralfunktion]
    
\end{definition}

\section{Anwendungen des Integrals}
Fläche zwischen zwei Graphen
\begin{theorem}[Fläche zwischen zwei Graphen]
    
\end{theorem}

Mittelwert einer Funktion: Motivieren durch diskrete Beispiele von arithmetischen Mittel => Verallgemeinerung auf unendlich viele unendlich feine gemittelte Werte
\begin{definition}[Mittelwert einer Funktion]
    
\end{definition}

\section{Uneigentliche Integrale}

\begin{definition}[Uneigentliches Integral]
    
\end{definition}

\section{Rotationskörper}
Einstieg mit vielen Beispielen (auch Gabriels Horn) => Volumen und FLächeninhalt bestimmen. 

Heuristische Herleitung des Volumens (Bohrmaschine). 

Volumen des Rotationskörpers zwischen zwei Funktionen: ACHTUNG, leicht falsch machen. 

\begin{definition}[Rotationskörper]
    
\end{definition}

\begin{theorem}[Volumen von Rotationskörpern]
    
\end{theorem}

\begin{proposition}[Volumen eines von zwei Funktionen begrenzten Rotationskörpers]
    
\end{proposition}
