\chapter{Kurvendiskussion}
Genug Tools in den vorausgegangenen Kapiteln gesammelt, um Funktionen auf einige Eigenschaften zu untersuchen. 

\section{Monotonie}
Definition von Monotonie

\begin{definition}[Monotonie]
    
\end{definition}

Monotoniesatz (über den Mittelwertsatz): Auch Gegenbeispiel für Umkehrung
\begin{theorem}[Monotoniesatz]
    
\end{theorem}

Graphisches und rechnerisches Bestimmen von Monotonie-Intervallen

\section{Krümmung}
Konvexe und konkave Funktionen, bzw. Funktionsbereiche (Rechts- und Linkskurven). 
\begin{definition}[Links-/Rechtskrümmung]
    
\end{definition}

Charakterisierung konvexer Funktionen liefert den Krümmungssatz!

Krümmungssatz: Wann liegt Rechtskurve vor, wann Linkskurve?
\begin{theorem}[Krümmungssatz]
    
\end{theorem}

\section{Extrempunkte, Sattelpunkte und Wendestellen}
Definition Lokale und Globale Extrema

\begin{definition}[Lokale/Globale Extrema]
    
\end{definition}

Satz: Notwendige Bedingung (Nullstellen der Ableitung => Vorzeichenwechsel in der Ableitung)
\begin{theorem}[Notwendige Bedingung für Extremstellen]
    
\end{theorem}

Satz: Hinreichende Bedingungen (Vorzeichen der zweiten Ableitung, Vorzeichenwechsel)
=> Beispiele für jeden Fall (Warum braucht man manchmal das Vorzeichenwechsel-Kriterium?)
\begin{theorem}[Hinreichende Bedingungen für Extremstellen]
    
\end{theorem}

Exkurs: Existenz von Maxima/Minima => Satz von Max/Min

Definition: Nur notwendige Bedingung erfüllt, aber kein Extrempunkt => Sattelpunkte
\begin{definition}[Sattelpunkt]
    
\end{definition}

Definition: Wendestellen
\begin{definition}[Wendestelle]
    
\end{definition}

Satz: Notwendige und hinreichende Bedingung für Wendestellen. 
\begin{theorem}[Notwendige und hinreichende Bedingung für Wendestellen]
    
\end{theorem}

NEW-Schema

\section{Asymptoten}
\begin{definition}[Waagerechte, senkrechte, schiefe, sonstige Asymptoten]
    
\end{definition}

Waagerechte Asymptoten und Senkrechte Asymptoten

Definition? 

Bestimmen an verschiedensten Beispielen: exp, log (Existenz), Gebrochenrationale Funktionen

\section{Symmetrie}
Definition: Achsensymmetrie (zu beliebiger Achse) => Herleitung anhand von Beispielen für Achsensymmetrie zu x-, bzw. y-Achse durch Manipulation des Graphen auf den einfachen Fall zurückführen!

\begin{definition}[Achsensymmetrische Funktionen]
    
\end{definition}

Definition: Punktsymmetrie (zu beliebigen Punkt) => Analog zu Achsensymmetrie Herleitung usw. 
\begin{definition}[Punktsymmetrische Funktionen]
    
\end{definition}

\section{Untersuchen von gebrochenrationalen Funktionen}
Polstellen, Definitionslücken und "`einfache"' Asymptoten. 

\section{Manipulieren von Graphen}
Wir haben bereits einige Male die Graphen von Funktionen auf bestimmte Weisen "manipuliert". Etwa, um einen "schwierigeren" Fall auf einen einfacheren Fall zurückzuführen (Siehe Symmetrien). 

\section{Extremwertprobleme}
Gehört nicht wirklich zur klassischen Kurvendiskussion, allerdings einfache Klausuraufgaben mit Extremwertbestimmung (Anwendung davon). 

Grundlegend zwei verschiedene Arten: Geometrische vs. funktionale Nebenbedingungen. 

VIELE Beispiele!
